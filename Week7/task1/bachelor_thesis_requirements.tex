\documentclass{report}
\usepackage{hyperref}
\usepackage[utf8]{inputenc}
\usepackage[danish]{babel}

\title{Bachelor thesis requirements}
\author{
    Andreas Zoega Vesterborg Vikke\\
    \texttt{cph-av105}
    \and
    Asger Hermind Sørensen\\
    \texttt{cph-as466}
    \and
    Martin Eli Frederiksen\\
    \texttt{cph-mf237}
    \and
    William Sehested Huusfeldt\\
    \texttt{cph-wh106}
}
\date{}
\begin{document}

\maketitle
\newpage


\tableofcontents
\newpage
\section{Indledning}
Jeg tager udgangspunkt i hvad der står i studieordningen for softwareudvikling på CPH Business,
fremtidigt referet til som studieordningen, og det kommer til at være den hovedanvendte kilde,
da den tager udgangspunkt i de krav der bliver sat til bachelor projektet for lige nøjagtigt 
denne uddannelse. Derudover kommer jeg også ind på hvad der bliver sat af krav til bachelor uddannelsen
som helhed ud fra hvad uddannelses- og forskningsministeriet skriver. 

\section{Introduktion til kravene}
Som skrevet i studieordningen\cite{studieordning}, 
skal bachelorprojektet kunne dokumentere den studerendes forståelse af praksis og centralt
anvendt teori og metode i relation til en praksisnær problemstilling, der tager udgangspunkt i en konkret 
opgave inden for uddannelsens område. Problemstillingen, der således skal være central for 
uddannelsen og erhvervet, formuleres af den studerende, eventualt i samarbejde med en privat eller offentlig virksomhed.
Dertil nævnes det at der er Cphbusiness der godkender problemstillingen. I følge studieordningen afvikles 
eksamen som en ekstern prøve, som sammen med prøven efter praktikken og uddannelsens øvrige prøver skal dokumentere, 
at uddannelsens mål for læringsudbytte er opnået. Prøven kan kun finde sted efter, at afsluttende prøve i 
praktikken og uddannelsens øvrige prøver er bestået.
Bachelor projektet finder sted på tredje semester og har et omfang af 15 ECTS point.
som beskrevet i studieordningen er læringsmålende følgende:
"Det afsluttende bachelorprojekt skal dokumentere, at uddannelsens
afgangsniveau er opnået, jf. bilag 1 i BEK for professionsbacheloruddannelsen i
softwareudvikling:\\

\noindent\textbf{\textit{Viden:}}\\
Den uddannede har viden om:
\begin{itemize}
    \item den strategiske rolle af test i systemudvikling
    \item globalisering af softwareproduktion
    \item systemarkitektur og forståelse af dens strategiske betydning for virksomhedens forretning
    \item anvendt teori og metode samt udbredte teknologier inden for domænet
    \item sammenhænge mellem anvendt teori, metode og teknologi og kan reflektere over disses egnethed i forskellige situationer
\end{itemize}
\noindent\textbf{\textit{Færdigheder:}}\\
Den uddannede kan:
\begin{itemize}
    \item håndtere planlægning og gennemførelse af test af større IT-systemer
    \item indgå professionelt i samarbejde omkring udvikling af store systemer ved anvendelse af udbredte metoder og teknologier
    \item sætte sig ind i nye teknologier og standarder til håndtering af integration mellem systemer
    \item gennem praksis udvikle egen kompetenceprofil fra primært at være en backend-udviklerprofil til at varetage opgaver som systemarkitekt
    \item håndtere fastlæggelse og realisering af en såvel forretningsmæssig som teknologisk hensigtsmæssig arkitektur for store systemer
\end{itemize}
\noindent\textbf{\textit{Kompetencer:}}\\
Den uddannede kan:
\begin{itemize}
    \item håndtere planlægning og gennemførelse af test af større IT-systemer
    \item indgå professionelt i samarbejde omkring udvikling af store systemer ved anvendelse af udbredte metoder og teknologier
    \item sætte sig ind i nye teknologier og standarder til håndtering af integration mellem systemer
    \item gennem praksis udvikle egen kompetenceprofil fra primært at være en backend-udviklerprofil til at varetage opgaver som systemarkitekt
    \item håndtere fastlæggelse og realisering af en såvel forretningsmæssig som teknologisk hensigtsmæssig arkitektur for store systemer
\end{itemize}

\noindent Bedømmelsen kommer af en mundtlig eksamen på bagrund af skriftlig arbejde.\\

\noindent\newline Jeg har taget kontakt til Jette Nielsen, souschef på Uddannelses- og forskningsministeriet angående hvilke krav 
de stiller uddannelser for bachelor projekter, og hvordan det i det hele taget foregår. 
Jette Nielsen har ikke svaret tilbage endnu.

\bibliographystyle{plain}
\bibliography{bachelor-references}

\end{document}